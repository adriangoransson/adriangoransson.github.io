\documentclass[a4paper]{article}

\usepackage[top=1.3cm, bottom=2cm, left=1.3cm, right=1.3cm]{geometry}
\usepackage[T1]{fontenc}
\usepackage[utf8]{inputenc}

\usepackage{xcolor}
\usepackage{lastpage}

\usepackage[hidelinks]{hyperref}

% Profile picture
\usepackage{wrapfig}
\usepackage{graphicx}
% Headers/footers
\usepackage{fancyhdr}

% Font
\usepackage{charter}

\pagestyle{fancy}

\renewcommand{\headrulewidth}{0pt}
\lfoot{\href{https://adriang.se}{https://adriang.se}}
\cfoot{\today}
\rfoot{$^\thepage\! / _{\pageref*{LastPage}}$}

\newcommand{\worktime}[1]{
    \hfill \begin{normalsize}#1\end{normalsize}
}

\newcommand{\school}[2]{\subsection*{#1 \worktime{#2}}}
\newcommand{\employment}[3]{\subsection*{#2 --- \textit{#1} \worktime{#3}}}

\newcommand{\heading}[1]{
    \pdfbookmark{#1}{#1} % necessary for unnumbered sections
    \section*{#1
        \medskip
        \hrule
    }
}

\newcommand{\shorturl}[1]{\href{https://#1}{\texttt{#1}}}

\begin{document}

\noindent
\Huge Adrian Göransson\\
\large \href{https://github.com/adriangoransson}{github.com/adriangoransson}
\hfill +46 709 22 67 19\\
\href{https://www.linkedin.com/in/adriangoransson}{linkedin.com/in/adriangoransson}
\hfill \href{mailto:adrian@adriang.se}{adrian@adriang.se}


\heading{Education}

    \school{Faculty of Engineering, Lund University}{Sep. 2017 -- \emph{est. Jan 2022}}
    Master of Science in Computer Science and Engineering at LTH.
    Relevant courses: concurrent programming, algorithms and data structures, compilers, computer communication, data security, functional programming and operating systems.

    \school{NTI-Gymnasiet Lund \textnormal{Network technician. Cisco CCNA certified.}}{2010 -- 2013}

\heading{Experience}

    \employment
        {Master Thesis student}
        {Axis Communications}
        {Sep. 2021 -- \emph{est. Jan 2022}}

    Evaluating the OLAP database ClickHouse for real-time analysis of big data and comparing its feature set and performance against ElasticSearch and Pandas/numpy.

    \employment
        {Software developer}
        {Axis Communications}
        {Oct. 2020 -- Aug 2021}

     Part-time job (Student worker) developing the tools used at Axis for in-device diagnostics. My job has been to optimize collection and inspection of crash dumps.

    \employment
        {Teaching assistant}
        {Lund University}
        {Sep. 2020 -- Oct. 2020}

    Teaching assistant for a course in concurrent programming (EDAP10) at LTH.
    This course was one of my favorites as a student and helping others get through it and understand the concepts introduced by the course was both challenging and fun.

    \employment
        {Co-founder, developer}
        {Bitiq}
        {Feb. 2018 -- Mar. 2019}

    Ticket sales platform for student hosted events in Lund. We built a serverless REST api in Go and used Postgres as a database. Web apps developed with Vue were used for purchases, event administration and ticket verification using QR scanning.

    \employment
        {Software developer}
        {\href{https://www.qlik.com/us/}{Qlik}}
        {Jun. 2018 -- Jul. 2018}

    Summer position where I helped create a proof of concept of a data visualisation tool to further enhance Qlik's data analytics service.
    Built with Vue and Qlik's chart library Picasso.js.

    \employment
        {DevOps}
        {\href{https://www.beepsend.com/}{Beepsend}
        \begin{small}(acquired by Twilio)\end{small}}
        {Jul. 2013 -- Jul. 2016}

    Beepsend, today Twilio Sweden, is a messaging company specialising in SMS. I was part of a small team building our internal and costumer facing REST APIs with PHP and web apps using CoffeeScript/Javascript (with Chaplin.js). Due to my networking background I was also in charge of managing connectivity over VPN and SS7/Sigtran. I helped set up CI pipelines using Jenkins and Docker.

\heading{Skills}

    \subsection*{Experienced}
    \begin{tabular}{rl}
        Languages & Javascript (Vue, React), Go, Python, Java, PHP, Rust, HTML, CSS, SQL, Posix shell.\\
        Databases & PostgresSQL, MySQL, SQLite.\\
        Tools & Git, sh (zsh, bash), Docker.
    \end{tabular}

    \subsection*{Intermediate}
    \begin{tabular}{rl}
        Languages & C, C\#, Scala, Kotlin, Haskell.\\
        Databases & ElasticSearch, ClickHouse.\\
        Other & AWS, Terraform.
    \end{tabular}

\heading{Open source projects}

    \begin{itemize}
        \item
        \href{https://github.com/adriangoransson/renamer}{\textbf{renamer} --- \textit{Rust}}\hfill \href{https://github.com/adriangoransson/renamer}{\texttt{github.com/adriangoransson/renamer}}

        Command line tool to rename multiple files at once. A port of the classic perl utility rename.pl written in Rust.

        \item
        \href{https://nationilund.se}{\textbf{Nation i Lund} --- \textit{Go}}\hfill \shorturl{nationilund.se}

        Clean, fast and easy listing of student nation events in Lund. Works by scraping the official studentlund website. Would later inspire the launch of Bitiq.

        \item \href{https://pomodoro-tracker.netlify.com}{\textbf{Pomodoro tracker} --- \textit{Javascript/Vue}} \hfill \shorturl{pomodoro-tracker.netlify.com}

        Simple but customisable Pomodoro (study technique) ``egg clock''. Works just as well on mobile and desktop. Available offline with service workers and has notification support.
    \end{itemize}

\bigskip
\center \large References available upon request

\end{document}
